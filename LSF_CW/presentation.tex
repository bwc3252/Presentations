% Author: Ben Champion <bwc3252@rit.edu>
% Main LaTeX file for Lenfest Community Weekend 2019 Presentation

\documentclass{beamer}

\usetheme{Amsterdam}

\usepackage{caption}

\begin{document}
\title{Astrophysical Parameter Inference with Gravitational Wave and Electromagnetic Data Channels}
\subtitle{Lenfest Community Weekend 2019}
\author{Ben Champion}
\institute{Rochester Institute of Technology \\ Center for Computational Relativity and Gravitation}

% logos
\titlegraphic{\includegraphics[width=6cm]{Images/RIT_RGB_hor_k}\hspace*{2cm}~%
   \includegraphics[width=2cm]{Images/ccrg_logo}
}

\date{
  June 8, 2019
}

\frame{
  \titlepage
}

\frame{ \frametitle{Table of contents}
  \tableofcontents
}

\section{Introduction}

\frame{ \frametitle{Classical Physics}
  \begin{itemize}
    \item Late 1600s - Newton's law of universal gravitation \pause
    \begin{figure}
      \includegraphics[width=1.75in]{Images/universal_gravitation}
      \caption*{\fontsize{8pt}{10pt}\selectfont Image by Dennis Nilsson}
    \end{figure} \pause
    \item "Action at a distance"
  \end{itemize}
}

\frame{ \frametitle{Special Relativity}
  \begin{columns}[c]
    \column{3.0in}
    \begin{itemize}
      \item 1905 - Einstein's theory of Special Relativity \pause
      \begin{itemize}
        \item The speed of light is independent of the motion of the observer \pause
        \item Consequences include length contraction, time dilation, etc. \pause
        \item Shortcoming: only applies to inertial (non-accelerating) reference frames \pause
      \end{itemize}
    \end{itemize}
    \column{1.5in}
    \begin{figure}
      \includegraphics[width=\linewidth]{Images/world_line}
      \caption*{\fontsize{8pt}{10pt}\selectfont Lightcone: the path through spacetime that a single flash of light, traveling in all directions, would take (image by K. Aainsqatsi)}
    \end{figure}
  \end{columns}
}

\frame{ \frametitle{Special Relativity}
  Classically, for an event with coordinates $(t, x)$ as measured by a stationary observer and $(t', x')$ as measured by an observer moving with velocity $v$:
  \begin{align*}
    t' &= t  \\
    x' &= x - vt
  \end{align*}

  \pause

  However, according to Special Relativity:
  \begin{align*}
    t' &= \gamma (t - \frac {vx} {c^2}) \\
    x' &= \gamma (x - vt) \\
    \text{where } \gamma &= \frac {1} {\sqrt{1 - \frac {v^2} {c^2}}}
  \end{align*} \pause
  This is called the Lorentz Transformation.
  }

\frame{ \frametitle{General Relativity}
  \begin{itemize}
    \item 1915 - Einstein's theory of General Relativity \pause
    \begin{itemize}
      \item Einstein's efforts to include acceleration and external forces (e.g. gravity) in relativity resulted in his theory of General Relativity \pause
      \item Objects with mass change the geometry of spacetime itself \pause
    \end{itemize}
    \item Einstein field equations: describe relationship between four-dimensional spacetime and and the energy-momentum contained in that spacetime \pause \\
    \begin{align*}
      R_{\mu v} - \frac {1} {2} R g_{\mu v} + \Lambda g_{\mu v} = \frac {8 \pi G} {c^4} T_{\mu v}
    \end{align*}
  \end{itemize}
}

\frame{ \frametitle{Einstein Field Equations}
  \begin{columns}[c]
    \column{2.5in}
    \begin{align*}
      R_{\mu \nu} - \frac {1} {2} R g_{\mu \nu} + \Lambda g_{\mu \nu} = \frac {8 \pi G} {c^4} T_{\mu \nu}
    \end{align*}
    \begin{tabular}{l l}
      $R_{\mu v}$   & Ricci curvature tensor \\
      $R$           & scalar curvature \\
      $g_{\mu \nu}$ & metric tensor \\
      $\Lambda$     & cosmological constant \\
      $T_{\mu \nu}$ & stress-energy tensor
    \end{tabular}
    \column{2.0in}
    \begin{figure}
      \includegraphics[width=\linewidth]{Images/general_relativity}
    \end{figure}
  \end{columns}
}

\frame{ \frametitle{Consequences of General Relativity}
  \begin{columns}[c]
    \column{2.5in}
    \begin{itemize}
      \item Bending of light around massive objects \pause
      \item Black holes \pause
      \item Gravitational time dilation \pause
      \item Gravitational waves \pause
    \end{itemize}
    \column{2.5in}
    \begin{figure}
      \includegraphics[width=\linewidth]{Images/lensing}
      \caption*{A galaxy bends the light from a more distant galaxy, resulting in \textit{gravitational lensing}}
    \end{figure}
  \end{columns}
}

\frame{ \frametitle{Gravitational Waves}
  \begin{itemize}
    \item Objects with mass cause distortions in spacetime \pause
    \item Motion involving a changing acceleration produces gravitational waves \pause
    \begin{itemize}
      \item (assuming this motion does not involve spherical or rotational symmetry) \pause
      \item (show animation) \pause
    \end{itemize}
    \item The gravitational waves produced by most objects are too small to detect \pause
    \item Gravitational waves produced by merging black holes or neutron stars are large enough to detect \pause
    \begin{itemize}
      \item Neutron stars are small, dense stars formed from the collapsed cores of giant stars
    \end{itemize}
  \end{itemize}
}

\frame{ \frametitle{LIGO}
  \begin{columns}[c]
    \column{3.0in}
    \begin{itemize}
      \item Laser Interferometer Gravitational Wave Observatory \pause
      \item Two detectors \pause
      \begin{itemize}
        \item Hanford, Washington \pause
        \item Livingston, Louisiana \pause
      \end{itemize}
      \item Designed to detect gravitational waves predicted by Einstein's theory of General Relativity \pause
      \item Over 1300 researchers from over 100 institutions \pause
    \end{itemize}
    \column{2.0in}
    \begin{figure}
      \includegraphics[width=\linewidth]{Images/ligo_livingston}
      \caption*{\fontsize{8pt}{10pt}\selectfont The Livingston detector (image by Caltech/MIT/LIGO Lab )}
    \end{figure}
\end{columns}
}

\frame{ \frametitle{Gravitational Wave Interferometry}
  \begin{columns}[c]
    \column{2.5in}
    \begin{itemize}
      \item LIGO has two detectors
    \end{itemize}
    \begin{figure}
      \includegraphics[width=2.0in]{Images/interferometer}
      \caption*{\fontsize{8pt}{10pt}\selectfont Image by Barry Barish}
    \end{figure}
    \pause
    \column{2.25in}
    \begin{itemize}
      \item As a gravitational wave passes, one arm stretches and the other shrinks \pause
      \item When the arms are the same length, the interference pattern detected is completely destructive \pause
      \item When one arm is longer than the other, it takes light more time to travel down that arm, resulting in a detectable interference pattern \pause
      \item (show animation)
    \end{itemize}
  \end{columns}
}

\frame{ \frametitle{Gravitational Wave Detections}
  \begin{figure}
    \includegraphics[width=\linewidth]{Images/detections}
    \caption*{\fontsize{8pt}{10pt}\selectfont Image by LIGO/Virgo/Georgia Tech/S. Ghonge \& K. Jani}
  \end{figure}
}

\section{Gravitational Wave Parameter Estimation}

\frame{ \frametitle{What is Parameter Estimation?}
  \begin{itemize}
    \item Estimation of astrophysical source parameters from observations \pause
    \begin{itemize}
      \item Intrinsic parameters: \pause
      \begin{itemize}
        \item Mass \pause
        \item Spin \pause
        \item Compactness (for neutron stars) \pause
      \end{itemize}
      \item Extrinsic parameters: \pause
      \begin{itemize}
        \item Sky location \pause
        \item Distance \pause
        \item Orientation \pause
      \end{itemize}
    \end{itemize}
  \item Goal is to recover a \textit{posterior distribution} of parameters, using theoretical models that give waveforms as functions of these parameters
  \end{itemize}
}

\frame{ \frametitle{Bayes' Theorem and Model Likelihoods}
  For a vector of unknown parameters $\vec \theta$, proposed waveform model $H$, and set of observations $\{d\}$, Bayes' Theorem gives the \textit{probability density function} of $\vec \theta$ as
  \begin{align*}
    p(\vec \theta | \{d\}, H) = \frac {p(\vec \theta | H) p(\{d\} | \vec \theta, H)} {P(\{d\} | H)}
  \end{align*} \pause
  \begin{itemize}
    \item $p(\vec \theta | H)$ is the \textit{prior} distribution of $\vec \theta$ \pause
    \item $p(\{d\} | \vec \theta, H)$ is the \textit{likelihood function} - the likelihood of observing $\{d\}$ given $\vec \theta$ and $H$ \pause
    \begin{itemize}
      \item Function of residuals - differences between model and data \pause
    \end{itemize}
    \item $P(\{d\} | H)$ is called the \textit{evidence} for a model (a constant for each model)
  \end{itemize}
}

\frame{ \frametitle{Adaptive Sampling}
  \begin{itemize}
    \item Sampling the parameter space - the possible values of $\vec \theta$ - and calculating the probability density at each point allows us to draw conclusions about the parameters \pause
    \item Problem: parameter spaces are high-dimensional and likelihood calculations are computationally expensive \pause
    \item Solution: Sample parameter space \textit{adaptively} - that is, focus on parts of the parameter space with high likelihoods
  \end{itemize}
}

\frame{ \frametitle{Posterior Distribution Example}
  \begin{figure}
    \includegraphics[width=2.5in]{Images/corner_example}
    \caption*{Example of posterior distribution of masses for GW150914 (arXiv:1807.10312)}
  \end{figure}
}

\section{Neutron Star Mergers and Kilonovae}

\frame{ \frametitle{Neutron Star Mergers and Kilonovae}
  \begin{columns}[c]
    \column{2.5in}
    \begin{itemize}
      \item Neutron stars are the smallest, densest stars \pause
      \item Binary neutron star (BNS) mergers produce gravitational waves \pause
      \item Kilonovae - supernova-like explosions resulting from BNS mergers - are a primary candidate for the creation of heavy elements (such as gold) in the universe \pause
    \end{itemize}
    \column{2.0in}
    \begin{figure}
      \includegraphics[width=2.0in]{Images/neutron_star}
      \caption*{A simulated view of a neutron star (image by Rafael Jean-Luc Alexandre)}
    \end{figure}
  \end{columns}
}

\frame{ \frametitle{Multimessenger Astrophysics}
  \begin{columns}[c]
    \column{2.5in}
    \begin{itemize}
      \item Kilonovae eject large amounts of material at high velocities \pause
      \item They produce electromagnetic (EM) transients through the radioactive decay of this ejected material \pause
      \item These EM signals can be observed and used in the analysis of BNS mergers \pause
      \item GW170817: first BNS merger detected by LIGO, accompanied by electromagnetic observations across frequency bands \pause
    \end{itemize}
    \column{2.25in}
    \begin{figure}
      \includegraphics[width=\linewidth]{Images/lightcurves}
      \caption*{GW170817 lightcurves (arXiv:1710.05840)}
    \end{figure}
    Lightcurve: light intensity as a function of time
  \end{columns}
}

\frame{ \frametitle{GW170817}
\begin{figure}
  \includegraphics[width=3.5in]{Images/timeline}
  \caption*{GW170817 timeline of detections (arXiv:1710.05833)}
\end{figure}
}

\section{GW/EM PE}

\frame{ \frametitle{EM Parameter Estimation}
  \begin{itemize}
    \item Parameter estimation using EM data channels is similar to gravitational wave parameter estimation \pause
    \item Likelihood function used for EM parameter estimation: the log-likelihood of parameter vector $\vec \theta$ is
    \begin{align*}
      \ln L = -0.5 \sum \frac {(x(t) - m(t | \vec \theta))^2} {\Delta x(t)^2 + \Delta m(t | \vec \theta)^2}
    \end{align*}
    where $x(t)$ is the lightcurve magnitude at time $t$, $m(t | \vec \theta)$ is the model value at time $t$ given the current model parameters, $\Delta x(t)$ is the data error, and $\Delta m(t | \vec \theta)$ is the model error.
  \end{itemize}
}

\frame{ \frametitle{EM Parameter Estimation}
  \begin{columns}
    \column{2.5in}
    \begin{figure}
      \includegraphics[width=2.5in]{Images/likelihood}
    \end{figure}
    \column{2.5in}
    \pause
    \begin{align*}
      \ln L = -0.5 \sum \frac {(x(t) - m(t | \vec \theta))^2} {\Delta x(t)^2 + \Delta m(t | \vec \theta)^2}
    \end{align*} \pause
    \begin{itemize}
      \item For a given set of parameters, run the model \pause
      \item Subtract data from model \pause
      \item Calculate log-likelihood from difference \pause
    \end{itemize}
  \end{columns}
}

\begin{frame}[fragile] \frametitle{\texttt{EM\_PE}: Rapid Parameter Estimation for EM Transients}
  \begin{columns}[c]
    \column{2.0in}
    \begin{itemize}
      \item Open-source \texttt{Python} package: \texttt{github.com/bwc3252/EM\_PE} \pause
      \item Parameter estimation for arbitrary models and data channels \pause
      \item Uses adaptive Monte Carlo integration to sample parameter space \pause
      \item Functionality: \pause
      \begin{itemize}
        \item Full PE \pause
        \item Visualization of results \pause
        \item Likelihood function for use in other PE codes \pause
      \end{itemize}
    \end{itemize}
    \column{3.0in}
    \begin{verbatim}
    from em_pe import sampler

    ### setup
    dat = "./"
    m = "two_comp"
    f = ["H.txt"]
    out = ""
    params = {...}

    s = sampler(dat, m, f, out)
    lnL = s.log_likelihood(params)
    \end{verbatim}
  \end{columns}
\end{frame}

\frame{ \frametitle{Example}
  \begin{columns}[c]
    \column{2.5in}
    \begin{figure}
      \includegraphics[width=2.2in]{Images/GW170817_woko2017.png}
      \caption*{Posterior distribution of parameters (ejecta mass and velocity) for GW170817}
    \end{figure}
    \column{2.5in}
    \begin{itemize}
      \item Kilonova model that takes ejecta mass and velocity as parameters \pause
      \item Model outputs lightcurve (light intensity as function of time) in various wavelength bands \pause
      \begin{itemize}
        \item $z$ band (900 nm wavelength) data used here \pause
      \end{itemize}
      \item Plot posterior probability (from Bayes' Theorem)
    \end{itemize}
  \end{columns}
}

%\frame{ \frametitle{Example}
%  \begin{columns}[c]
%    \column{2.5in}
%    \begin{figure}
%      \includegraphics[width=2.6in]{Images/GW170817_me2017_mej.png}
%    \end{figure}
%    \column{2.5in}
%    \begin{itemize}
%      \item Two-component kilonova model (two ejecta masses and velocities)\pause
%      \begin{itemize}
%        \item Different opacity for each \pause
%      \end{itemize}
%      \item Same data band ($z$) used here \pause
%      \item This model also has $v_{ej}$ for each component as parameters, but only $m_{ej}$ is shown here
%    \end{itemize}
%  \end{columns}
%}

\frame{ \frametitle{Ongoing Work}
  \begin{itemize}
    \item Joint parameter estimation: PE using both gravitational wave and electromagnetic data \pause
    \begin{itemize}
      \item Combined likelihood: $\ln L = \ln L_{GW} + \ln L_{EM}$ \pause
      \item Different data channels == different constraints on parameters \pause
    \end{itemize}
    \item Combining parameters \pause
    \begin{itemize}
      \item Kilonova ejecta parameters in terms of BNS parameters \pause
    \end{itemize}
    \item Adding new and more complex models \pause
    \item Optimizing PE codes \pause
    \begin{itemize}
      \item GPUs, parallel computing, ...
    \end{itemize}
  \end{itemize}
}

\section{Conclusion}

\frame{
  Questions?
  \begin{figure}
    \includegraphics[width=\linewidth]{Images/gw}
    \caption*{Image from R. Hurt/Caltech-JPL}
  \end{figure}
}

\end{document}
